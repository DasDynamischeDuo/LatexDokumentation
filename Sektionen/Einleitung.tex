\section{Einleitung}

\subsection{Vorstellung des Prokeltes}

Unser Projekt das Digital-Input Computer-Keyboard, ist wie der Name schon vermuten lässt ein 
Digitales Keyboard. Man kann damit mithilfe seines PCs und einer Tastatur Keyboard spielen. Es kann 
allerdings auch noch mehr. Der Benutzer hat die Möglichkeit verschiedene Samples ( Tonspuren die 
Instrumente darstellen ) einzustellen. Somit kann er theoretisch jedes Musikinstrument über seine 
Tastatur spielen. Außerdem kann der Benutzer gespieltes aufnehmen und abspielen lassen. Somit kann 
er mit verschiedenen Instrumenten und Tonspuren eigene Lieder zusammensampeln. 

Dieses Projekt hat Emanuel und mir sehr gut gefallen, da wir beide ein Musikinstrument spielen und 
auch allgemein Musik begeistert sind. Ein eigenes kleines Sample-Programm zu gestalten lag also 
nahe. \\



\subsubsection{Unser Team}

Bei unserem IT-Projekt hatten wir die Möglichkeit Teams zu bilden um größere Projekte umzusetzen. Das Team des Projektes Digital Input Computer Keyboard, besteht aus Emanuel Hubenschmid und Fabian Zeller. Diese beiden sind musikalisch aktiv und hatten somit Interesse daran den Computer mit musikalischer Kreativität zu verbinden. 
Das Arbeitsklima untereinander war durchgehend angenehm, da jeder seine Arbeit hatte, diese erledigt hat und alles dann gemeinsam zusammengeführt wurde. Es kam also nie zu Streit oder Uneinigkeit über den weiteren Verlauf des Projekts. Da wir verschiedene Stärken haben haben wir unser Projekt auch entsprechend aufgeteilt. Emanuel hat sich größtenteils um Wiedergabe von Samples gekümmert, während Fabian die GUI entsprechend angepasst hat.

\newpage


\subsection{Das Projektmanagement}
\subsubsection{SCRUM}


\textbf{Definition:}\\
Scrum wurde ursprünglich in der Softwaretechnik entwickelt, ist aber auch in anderen Bereichen einsetzbar. Mittlerweile wird Scrum in sehr vielen Firmen eingesetzt. Scrum basiert darauf das Projekt oder auch die Projektplanung kontinuierlich zu verbessern. Es ist also sehr flexibel und passt sich ständig an. Ziel von dieser kontinuierlichen Verbesserung ist die schnelle und kostengünstige Entwicklung hochwertiger Produkte. \\
Anfangs hat man nur eine Vorstellung seines Projekts die durch den Kunden oder den Entwickler gegeben wird. Man versucht dann diese Vorstellung anhand von einer Liste mit Eigenschaften festzuhalten. Anders als bei anderen Methoden fertigt man kein Lasten- oder Pflichtenheft an, sondern eben jene Liste. Die Liste dieser Eigenschaften oder auch Anforderungen an das Endprodukt nennt man Backlog.\\
Diese Anforderungen werden dann intervallartig abgearbeitet. Diese Intervallartigen umsetzungen werden Sprints genannt. Nach Abschluss eines Sprints hat man dann also ein fertiges Teilprodukt des ganzen, dass man testen oder dem Kunden präsentieren kann. Hier fallen dann schon eventuelle Änderungen auf,die dann in das Backlog, bzw. in das nächste Sprint einfließen. Somit verändert sich dass Backlog ständig und das Endprodukt wird optimiert.\\
Zusammengefasst basiert Scrum auf 3 "Säulen":

\begin{itemize}
\item[•] \textbf{Transparenz:} Der Fortschritt und die Verzögerungen werden kontinuierlich festgehalten und man hat immer eine Vorstellung davon wie weit das Projekt fortgeschritten ist.
\item[•] \textbf{Überprüfung:} Durch die Sprints werden regelmäßig Teilprodukte geliefert die man testen, überprüfen und optimieren kann. 
\item[•] \textbf{Anpassung:} Dadurch dass nicht von Anfang das Endprodukt festgelegt ist, sondern kontinuierlich angepasst wird, entstehen aus einem meist komplexen System viele kleinere Teilsysteme die einfacher zu bewältigen sind. 
\end{itemize}

Wen dieses Konzept interessiert kann sich hier weiter über Scrum erkundigen: Link\\

\textbf{Unsere Erfahrung mit Scrum}\\

Wir haben für unser Schulprojekt Scrum verwendet. Wir haben ein Backlog erstellt und wöchentlich Sprints abgeschickt. Scrum hat uns sehr geholfen unser Projekt zu strukturieren, allerdings waren wöchentliche Sprints nicht optimal. Da wir als Schüler nicht Vollzeit an unserem Projekt arbeiten und die Sprints wöchentlich abzuschicken waren, haben sich manche Stories über mehrere Sprints angestaut. In einem Betrieb oder einer Entwicklerfirma in welcher nur an einem Projekt entwickelt wird sind häufige Sprints sinnvoll, bei nur 2 Leuten und eine quasi Nebenbeschäftigung waren diese doch in zu kurzen Intervallen. Ansonsten war Scrum eine große Hilfe.




\subsubsection{GitHub} \label{sssec:GitHub}
"Build software better, together." (Motto von GitHub)\\

Für unsere Projektarbeiten verwendeten wir GitHub. GitHub ist ein webbasierter Hosting-Dienst für 
Softwareprojekte. Damit Emanuel und ich also bequem von zuhause aus zusammen arbeiten konnten haben 
wir uns ein Repository in GitHub eingerichtet. Zusammen mit dem Eclipse-Plugin EGit konnten wir 
unsere Arbeit austauschen und vergleichen. Obwohl es Anfangs Probleme mit der Bedienung und den 
verschiedenen Funktionen GitHubs und EGits gab, hat es uns doch sehr geholfen und vieles 
vereinfacht. So konnten wir zum Beispiel gleichzeitig an verschieden Problemen arbeiten, indem wir 
verschiedene Branches ( "Pfade" also Ableger des Projekts ) erstellt haben und dann an diesen 
Branches gearbeitet haben. Sobald dann ein Problem behoben war hat man den Pfad wieder dem 
Hauptprojekt hinzugefügt und konnte besprechen was genau gemacht wurde und was vielleicht noch 
verbessert werden muss.\\

GitHub war uns im allgemeinen eine sehr große Hilfe, aber zu Anfang auch eine große Hürde. Bis wir 
zurechtkamen mit den Branches, Commits ect. hat es eine Weile gedauert. Man kann aber durchaus 
behaupten das sich der Aufwand gelohnt hat. Ich würde jedem der ein Softwareprojekt entwickelt 
empfehlen mit GitHub zu arbeiten, auch wenn er alleine daran arbeitet. Denn EGit zwingt einen dazu 
seine Änderungen zu dokumentieren und zwischenzuspeichern. Dies ist zwar etwas nervig, aber man 
kann 
sein Projekt immer wieder auf einen beliebigen Standpunkt zurücksetzten wenn etwas komplett 
schiefgelaufen ist. Dies hat mir oft sehr viel Arbeit erspart.\\

Gerade jetzt für diese Dokumentation verwenden wir auch GitHub for Windows. Ein Programm für 
Windows 
welches einem erlaubt jede Art von Datei über GitHub zu veröffentlichen und zu bearbeiten. GitHub 
hat uns also sehr viel Arbeit erspart und es uns ermöglicht obwohl wir weit auseinander wohnen ein 
Projekt auch zusammen zu erarbeiten.\\


